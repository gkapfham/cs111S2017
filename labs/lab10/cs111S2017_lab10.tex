% CS 111 style
% Typical usage (all UPPERCASE items are optional):
%       \input 111pre
%       \begin{document}
%       \MYTITLE{Title of document, e.g., Lab 1\\Due ...}
%       \MYHEADERS{short title}{other running head, e.g., due date}
%       \PURPOSE{Description of purpose}
%       \SUMMARY{Very short overview of assignment}
%       \DETAILS{Detailed description}
%         \SUBHEAD{if needed} ...
%         \SUBHEAD{if needed} ...
%          ...
%       \HANDIN{What to hand in and how}
%       \begin{checklist}
%       \item ...
%       \end{checklist}
% There is no need to include a "\documentstyle."
% However, there should be an "\end{document}."
%
%===========================================================
\documentclass[11pt,twoside,titlepage]{article}
%%NEED TO ADD epsf!!
\usepackage{threeparttop}
\usepackage{graphicx}
\usepackage{latexsym}
\usepackage{color}
\usepackage{listings}
\usepackage{fancyvrb}
%\usepackage{pgf,pgfarrows,pgfnodes,pgfautomata,pgfheaps,pgfshade}
\usepackage{tikz}
\usepackage[normalem]{ulem}
\tikzset{
    %Define standard arrow tip
%    >=stealth',
    %Define style for boxes
    oval/.style={
           rectangle,
           rounded corners,
           draw=black, very thick,
           text width=6.5em,
           minimum height=2em,
           text centered},
    % Define arrow style
    arr/.style={
           ->,
           thick,
           shorten <=2pt,
           shorten >=2pt,}
}
\usepackage[noend]{algorithmic}
\usepackage[noend]{algorithm}
\newcommand{\bfor}{{\bf for\ }}
\newcommand{\bthen}{{\bf then\ }}
\newcommand{\bwhile}{{\bf while\ }}
\newcommand{\btrue}{{\bf true\ }}
\newcommand{\bfalse}{{\bf false\ }}
\newcommand{\bto}{{\bf to\ }}
\newcommand{\bdo}{{\bf do\ }}
\newcommand{\bif}{{\bf if\ }}
\newcommand{\belse}{{\bf else\ }}
\newcommand{\band}{{\bf and\ }}
\newcommand{\breturn}{{\bf return\ }}
\newcommand{\mod}{{\rm mod}}
\renewcommand{\algorithmiccomment}[1]{$\rhd$ #1}
\newenvironment{checklist}{\par\noindent\hspace{-.25in}{\bf Checklist:}\renewcommand{\labelitemi}{$\Box$}%
\begin{itemize}}{\end{itemize}}
\pagestyle{threepartheadings}
\usepackage{url}
\usepackage{wrapfig}
% \usepackage{hyperref}
\usepackage[hidelinks]{hyperref}
%=========================
% One-inch margins everywhere
%=========================
\setlength{\topmargin}{0in}
\setlength{\textheight}{8.5in}
\setlength{\oddsidemargin}{0in}
\setlength{\evensidemargin}{0in}
\setlength{\textwidth}{6.5in}
%===============================
%===============================
% Macro for document title:
%===============================
\newcommand{\MYTITLE}[1]%
   {\begin{center}
     \begin{center}
     \bf
     CMPSC 111\\Introduction to Computer Science I\\
     Fall 2016\\
     \medskip
     \end{center}
     \bf
     #1
     \end{center}
}
%================================
% Macro for headings:
%================================
\newcommand{\MYHEADERS}[2]%
   {\lhead{#1}
    \rhead{#2}
    \immediate\write16{}
    \immediate\write16{DATE OF HANDOUT?}
    \read16 to \dateofhandout
    \lfoot{\sc Handed out on \dateofhandout}
    \immediate\write16{}
    \immediate\write16{HANDOUT NUMBER?}
    \read16 to\handoutnum
    \rfoot{Handout \handoutnum}
   }

%================================
% Macro for bold italic:
%================================
\newcommand{\bit}[1]{{\textit{\textbf{#1}}}}

%=========================
% Non-zero paragraph skips.
%=========================
\setlength{\parskip}{1ex}

%=========================
% Create various environments:
%=========================
\newcommand{\PURPOSE}{\par\noindent\hspace{-.25in}{\bf Purpose:\ }}
\newcommand{\SUMMARY}{\par\noindent\hspace{-.25in}{\bf Summary:\ }}
\newcommand{\DETAILS}{\par\noindent\hspace{-.25in}{\bf Details:\ }}
\newcommand{\HANDIN}{\par\noindent\hspace{-.25in}{\bf Hand in:\ }}
\newcommand{\SUBHEAD}[1]{\bigskip\par\noindent\hspace{-.1in}{\sc #1}\\}
%\newenvironment{CHECKLIST}{\begin{itemize}}{\end{itemize}}

\begin{document}

\MYTITLE{Lab 10\\Assigned: April 6, 2017\\ Due: April 13, 2017 by 2:30pm}

\vspace{-0.2in}
\subsection*{Objectives}
\vspace{-0.05in}

To learn more about how to use iteration constructs, such as the {\tt while} and {\tt for} loops, when writing Java
programs. In addition, to explore some of the advanced graphical features that Java provides. Finally, to learn how to
conduct an empirical study that uses operating system timers to evaluate how the number of loop iterations influences
the execution time of a Java program.

\vspace{-0.15in}
\subsection*{General Guidelines for Labs}
\vspace{-0.05in}

\begin{itemize}

\item {\bf Work on the Alden Hall computers.} If you want to work on a different machine, be sure to transfer your
  programs to the Alden machines and re-run them before submitting.

\item {\bf Update your repository often!} You should {\tt add}, {\tt commit}, and {\tt push} your updated files each
  time you work on them.  I will not grade your programs until the due date has passed.

\item {\bf Review the Honor Code policy.} You may discuss programs with others, but programs that are nearly identical
  to others' will be taken as evidence of violating the Honor Code.

\end{itemize}

\vspace{-0.25in}
\subsection*{Reading Assignment}
\vspace{-0.05in}

Review Section 6.4 to learn more about the structure and behavior of the {\tt for} loop construct provided by the Java
programming language, paying close attention to the three key parts of the {\tt for} loop's declaration.  Please also
review Section 5.4 to refresh your knowledge of the {\tt while} loop and study Sections 11.1 through 11.6 to learn about
the concepts associated with exception handling. To learn more about fractals, and the Mandelbrot set that we will
visualize in this laboratory assignment, please study the following Web site available at
\url{http://jonisalonen.com/2013/lets-draw-the-mandelbrot-set}.  Students who have never explored the concept of
fractals are also encouraged to review \url{http://en.wikipedia.org/wiki/Mandelbrot_set}.

\vspace{-0.1in}
\subsection*{Creating a New Directory and Starting the Project}
\vspace{-0.05in}

After changing into the ``{\tt cs111F2016-share/}'' directory, which contains our course's version control repository,
you should type the command ``{\tt git pull}'' to download the source code for this laboratory assignment.  In your own
``{\tt cs111F2016-<your user name>}'' repository inside the ``{\tt labs/}'' directory, create a directory called ``{\tt
  lab10}''. Using the method described in a previous laboratory assignment, please copy the source code from the share
repository to your own repository. Now, change into the ``{\tt labs/lab10/}'' directory, in your own Git repository, and
use ``{\tt gvim}'' to study the source code of the provided files. What methods do these classes provide? How do they
work? While you do not need to understand the details of the provided source code, you should be able to add explanatory
comments that highlight the basic points about how this program operates. Students who do not understand these two programs
should ask the instructor for assistance.

\vspace{-0.1in}
\subsection*{Understanding and Empirically Evaluating a Fractal Generator}
\vspace{-0.05in}

Once you have developed a basic understanding of the {\tt MandelbrotBW.java} and {\tt MandelbrotColor.java} files, you
should try to compile and run these programs. Please notice that these programs do not produce any output in the
terminal window.  Instead, you should see that they create a graphics file in the same directory where they are stored.
For instance, if you run the {\tt MandelbrotColor} program you will see that it creates a {\tt mandelbrot-color.png}
file.  If you want to view the contents of this file, you can type the command ``{\tt xdg-open mandelbrot-color.png}''
in your terminal window.  What do you now see on the screen? What are the characteristics of this image?

Please use {\tt gvim} to display and edit the source code of the {\tt MandelbrotColor.java} file, noticing that it
declares an {\tt int} variable called {\tt max} and initializes it to the value of $1000$. How does the value of this
variable influence the time taken to create the Mandelbrot graphic? To answer this question, you can change the value of
max to, for instance, $100$, and recompile and run ``{\tt /usr/bin/time java MandelbrotColor}'' in your terminal.
After running this command, the number that is postfixed with the ``{\tt user}'' label will give the amount of time need
to create the fractal. As part of this assignment, you should set the {\tt max} variable to take on the values of $10,
100, 1000, \mbox{and}, \- 10000$, timing the execution with {\tt /usr/bin/time} and using {\tt xdg-open} to see how this
changes the resulting visualization. Along with recording the execution time for each program configuration, you should
save each distinct graphic with a unique name and upload it to your Bitbucket repository.

Now, find the line in the {\tt MandelbrotColor.java} file that is written in the following fashion:

{\tt colors[i] = Color.HSBtoRGB(i/256f, 1, i/(i+8f));}

\noindent This line of code configures the way in which {\tt MandelbrotColor} will display the colors in the fractal.
What happens when you change the value of ``{\tt 256f}'' to a different floating point value? Will changing this value
influence the efficiency of the fractal generator? To answer these questions you should set this value to the values of
$32, 64, 128, 256, 512, \mbox{and}, 1024$---while keeping the value of {\tt max} set to the default of $1000$---and then
again use {\tt /usr/bin/time} to measure the program's performance. In addition to recording the running time for each
program configuration, you should save each graphic with a unique name in your repository.  As you vary both this value
and the value of the {\tt max} variable, you should determine how execution time varies and how the image changes.

\vspace{-0.15in}
\subsection*{Required Deliverables}
\vspace{-0.05in}

For this assignment you should submit versions of the following deliverables through both the Bitbucket
repository and in a signed and printed format; do not print the color Mandelbrot graphics.

\vspace{-0.125in}
\begin{enumerate}
    \setlength{\itemsep}{0pt}

  \item Completed, fully commented, and properly formatted versions of the two source code files.
  \item The black and white version of the Mandelbrot graphic, saved with the default file name.
  \item Appropriately named versions of all of the color versions of the Mandelbrot graphics.
  \item Data tables that show the performance timings associated with varying the two parameters.
  \item A written report, saved in {\tt time}, explaining the trends in your Mandelbrot data tables.
  \item A written report, saved in {\tt color}, discussing how the Mandelbrot graphic changes.
  \item A written reflection, saved in {\tt reflection}, about the challenges you faced during this lab.

\end{enumerate}
\vspace{-0.1in}

Please make sure that you use the appropriate {\tt git} commands to save all of these required deliverables in the
``{\tt lab10/}'' directory of your own Git repository that is hosted by Bitbucket.

\end{document}


